\section{EAGLE}
Die Software \ac{eagle} von Autodesk ist eine \ac{cad} Software zur Entwicklung von Leiterplatten (\ac{pcb}).
Bis 2016 hat das deutsche Unternehmen CadSoft Computer die Software entwickelt, angefangen im Jahr 1988 mit dem Erscheinen von Eagle, damals noch eine 16 Bit Anwendung für DOS.
Ab 2000 wurde die Unterstützung für DOS niedergelegt und die Software wurde für Windows und Linux weiterentwickelt.
Bis 2016 wurde die Software für 64 Bit entwickelt und neue Funktionen wurden hinzugefügt.
Ab 2016 ist die Software im Besitz von Autodesk, die den einmaligen Kauf der Software auf ein Abo Modell umgestellt haben.\par

Eagle besteht grundsätzlich aus zwei Fenstern, zum einen die schematische bzw. Schaltplan-Darstellung und zum anderen der Leiterplattendesigner.
In der Schaltplan-Darstellung wird dieser erstellt, die einzelnen Bauteile definiert und deren Eigenschaften festgelegt.
Im Leiterplattendesigner wird die eigentliche Platine erstellt, die Bauteile auf der Platine platziert und mit Leiterbahnen verbunden.
\subsection{Versionsvergleich}
% Table generated by Excel2LaTeX from sheet 'Tabelle1'
\begin{table}[htbp]
	\centering
	% Table generated by Excel2LaTeX from sheet 'Tabelle1'
\begin{tabular}{lcccc}
\toprule
\multirow{2}[2]{*}{\textbf{Version}} & \multirow{2}[2]{*}{Premium} & Student \& & \multirow{2}[2]{*}{Standard} & \multirow{2}[2]{*}{Free} \\
\multicolumn{1}{l}{} & \multicolumn{1}{c}{} & educator & \multicolumn{1}{c}{} & \multicolumn{1}{c}{} \\
\midrule
\textbf{Schaltpläne pro Projekt} & \multicolumn{1}{c}{999} & \multicolumn{1}{c}{999} & \multicolumn{1}{c}{99} & \multicolumn{1}{c}{2} \\
\midrule
\textbf{Zahl der Schichten} & \multicolumn{1}{c}{16} & \multicolumn{1}{c}{16} & \multicolumn{1}{c}{4} & \multicolumn{1}{c}{2} \\
\midrule
\textbf{Größe der Leiterplatte} & 4 m²  & 4 m²  & 160 cm² & 80 cm² \\
\midrule
\multirow{2}[2]{*}{\textbf{Nutzung}} & \multirow{2}[2]{*}{beliebig} & \multirow{2}[2]{*}{Ausbildung} & \multirow{2}[2]{*}{beliebig} & privat \& nicht- \\
\multicolumn{1}{l}{} & \multicolumn{1}{c}{} & \multicolumn{1}{c}{} & \multicolumn{1}{c}{} & kommerziell \\
\midrule
\textbf{Kosten pro Monat} & USD 65 & gratis & USD 15 & gratis \\
\midrule
\textbf{Kosten pro Jahr} & USD 500 & gratis & USD 100 & gratis \\
\bottomrule
\end{tabular}%

	\caption{Verfügbare \ac{eagle}-Versionen ab 2017 \cite[vgl.][]{autodesk-eagle}}
\end{table}

\subsection{Entwicklung mit EAGLE}
\subsubsection{Schaltplan}
Zuserst wird in \ac{eagle} der Schaltplan erstellt.
Dabei werden alle benötigten Komponenten eingefügt und miteinander verbunden.
Eagle verfügt bereits über eine große Bibliothek an Komponenten von verschiedenen Herstellern, jedoch müssen für speziellere Komponenten Bibliotheken gesucht werden oder selbst erstellt werden.
Eagle unterstützt die Erstellung und Bearbeitung von Bibliotheken.
Diese Bibliotheken können anschließend direkt in ein Projekt eingebunden werden.\par

Nachdem der Schaltplan erstellt wurde, kann die Schaltung mittels \ac{erc} überprüft werden.
Diese Funktion überprüft, ob alle grundlegenden Anforderungen einer Schaltung erfüllt wurden, z.B. ob alle Anschlüsse verbunden worden sind oder ob alle Bauteile einen Wert haben.

\subsubsection{Layout}
Im Layout-Fenster wird die eigentliche Platine erstellt.
Um das Platinen Design Fenster zu öffnen, wird der Befehl Board in der Schaltplanansicht eingegeben.
Alle Komponenten aus dem Schaltplan werden übernommen, und die Verbindungen der Komponenten werden durch gelbe Linien gekennzeichnet.\par

Zuerst werden die Bauteile positioniert. Dabei werden alle Komponenten, die miteinander verbunden werden müssen, nahe beieinander platziert.
Nachdem alle Teile platziert wurden, können die Leiterbahnen eingezeichnet werden.
Es kann per Hand mit dem Befehl \enquote{route} gezeichnet werden, oder mittels Autorouter erstellt werden.
Der Autorouter benötigt jedoch viel Erfahrung, um korrekt genutzt werden zu können und das Ergebnis ist meist nicht ideal.
Mit Hand zeichnen ist am Anfang deutlich einfacher.\par

Nachdem die Leiterbahnen erstellt wurden wird die Platine mit dem \ac{drc} geprüft werden.
Dieser Befehl prüft wiederrum, ob alle grundlegenden Anforderungen erfüllt wurden, z.B. Mindestabstand der Leiterbahnen eingehalten oder noch nicht vollständig geroutete Leiterbahnen vorhanden sind.