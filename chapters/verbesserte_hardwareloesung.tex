\section{Konzept}
Die neue Lösung sollte den Aufbau einer Station vereinfachen und die Ausfallsrate einer Station auf gänzlich null reduzieren.
Um dieses Ziel zu erreichen, werden folgende Ansätze definiert:

\paragraph{Zusammenfassen der Platinen zu einer einzigen:}
Die Reduktion der Anzahl der Platinen auf eine Platine ermöglicht die Zusammenfassung von verschiedenen Modulen.
Damit werden die Produktionskosten gesenkt und die Montage in den Stationen vereinfacht.

\paragraph{Reduzierung der Bauteile:}
Die Zusammenfassung der Module auf eine einzelne Platine ermöglicht die Reduktion von Bauteilen.
Insbesondere werden Bauteile für die Verbindungstechnik eingespart.
Dies führt zu einer Kostenreduktion bei gleichzeitiger Senkung der Fehlermöglichkeiten. 

\paragraph{Entfall der Verdrahtung zwischen den Einzelplatinen:}
Der Entfall der Verdrahtung führt zu einer Steigerung der Übersichtlichkeit und reduziert zusätzlich den Platzbedarf.
Aufwände für die händische Verdrahtung und die Prüfung der Verbindungen der einzelnen Platinen entfällt.
Fehlverdrahtung wird konstruktiv durch die Einzelplatine verhindert.

\paragraph{Verbesserung der Spannungsversorgung:}
Leistungsbeschränkungen sowie die Verwendung eines neuen verbesserten Spannungsreglers verbessern die Spannungsqualität.
Die verstärkte Spannungsversorgung ermöglicht ein späteres Update auf den leistungsstärkeren \ac{rpi} Model 4.

\paragraph{Einführung WatchDogTimer:}
Zusätzlich wird ein Watchdog mittels eines Mikroprozessors verbaut.
Dieser wird benötigt, um eventuelle Ausfälle des \ac{rpi} und somit der gesamten Station zu erkennen.
Falls ein Fehler auftritt, setzt der Mikroprozessor den \ac{rpi} zurück und die Station startet neu.
Nachdem die Station neugestartet ist, wird der Normalbetrieb wieder aufgenommen und der Watchdog nimmt den Zustand des Überwachens wieder ein.

%Lösungen für probleme, zusätzliche vorteile
\section{Spannungsversorgung}
\subsection{Anforderungen}
Um einen möglichst reibungslosen Betrieb zu garantieren, werden an die Spannungsversorgung mehrere Anforderungen gestellt:
\begin{itemize}
	\item Der Eingangsspannungsbereich muss mindestens 24 bis 35 V betragen.
	\item Die gewünschte Ausgangsspannung beträgt 5 V.
	\item Der Oberwellenanteil (ripple) der Ausgangsspannung soll unter 1\% liegen. 
	\item Die Strombelastbarkeit muss mindestens 2.5 A betragen.
\end{itemize}
Die Versorgungsspannung wird über  \ac{poe} (Power over Ethernet) zur Verfügung gestellt.
Der Spannungsregler muss diese Art der Versorgungsspannung unterstützen.
Der Spannungsregler sollte die 24 V auf 5 V möglichst verlustarm reduzieren.
Alle versorgten Komponenten auf der Platine arbeiten mit einer Spannung von 5 V.
Diese Komponenten sind z.B. der \ac{rpi}, der Mikrocontroller und die Verstärkerschaltung für Mikrofon und Lautsprecher.\par

Die Spannungsversorgung soll bei Spannungsschwankungen kurzzeitig die Versorgung aufrechterhalten, um dem Versagen einer Station vorzubeugen.
Zur Stützung der Spannungsversorgung werden Kondensatoren auf der Eingangsspannungsseite und auf der Ausgangsspannungsseite benötigt.

\subsection{Auswahl Spannungsregler}
Aufgrund der oben angeführten Anforderungen wird der Spannungsregler LM33630 ausgewählt.
Dieser Spannungsregler kann im Bereich von 3.8 V bis 36 V betrieben werden.
Die Ausgangsspannung ist immer kleiner als die Eingangsspannung und liegt im Bereich von 1 V bis 24 V.
Der LM33630 ist ein Step-down Spannungsregler, das bedeutet er kann nur von einer höheren auf eine niedrigere Spannung regeln.\par

Um eine Schwingung, die wenig Aufwand zur Glättung benötigt, zu erzeugen, verwendet der Spannungsregler eine hohe Schaltfrequenz von 2.1 MHz.
Der LM33630 hat standardmäßig einen maximalen Ausgangsstrom von 3 A, dieser wird auch nahezu vollständig benötigt für den \ac{rpi}, der allein bereits ca. 1.5 A bezieht.
Der Lautsprecher mit der Vorschaltung benötigt ebenfalls über 0.5 A.\par

Vom LM33630 sind mehrere Varianten verfügbar, es gibt die DDA und RNX Variante.
Die DDA Variante hat 8 Pins und einen eingebauten Kühlkörper, der mit einem großen Massefeld verbunden werden kann, damit der \ac{ic} nicht überhitzt.
Die RNX Variante hat 12 Pins und anstatt eines größeren Kühlfläche mehrere Ground Pins, die ebenfalls zur Kühlung des \ac{ic} dienen.
Zusätzlich gibt es eine weitere Unterteilung der zwei Varianten: es gibt verschiedene Schaltfrequenzen, nämlich 400, 1400 und 2100 kHz.\par
Die technischen Daten dieses Spannungsreglers sind in vollem Umfang im Datenblatt ersichtlich. \cite[vgl.][]{lmr33630-datasheet}

\subsection{Funktion des Spannungsreglers}
Der hier gewählte Spannungsregler ist ein Schaltregler, d.h. er verändert den Effektivwert der Spannung mit \ac{pwm}.
Über einen Spannungsteiler kann die gewünschte Ausgangsspannung eingestellt werden.
Weiters kann der Spannungsregler abgeschaltet werden, wenn 0 V am Enable Pin angelegt werden.
Dies sorgt für eine komplette Abschaltung der Steuerlogik.
Die Verwendung dieser Funktion ist jedoch nicht vorgesehen.

\subsection{Zusatzbeschaltung des Spannungsreglers}
Der Spannungsregler ist universell einsetzbar, weshalb die genauen Betriebseigenschaften über die externe Beschaltung festgelegt werden.
\paragraph{Eingangskondensator:}
Auf der Eingangsseite, d.h. der 24 V Seite wird ein 820 µF Kondensator verbaut.
Dieser dient in erster Linie zum Abfangen und Ausgleichen von Spannungsschwankungen.
Diese Spannungsschwankungen können aufgrund von sich schlagartig änderndem Leistungsverbrauch entstehen.

\paragraph{Ausgangskondensator:}
Nach dem Spannungsregler auf der 5 V Seite werden zwei Kondensatoren mit jeweils der Kapazität von 1 mF verbaut, das ergibt eine Gesamtkapazität auf der 5 V Seite von 2 mF.
Diese werden zur nahezu vollständigen Glättung der Ausgangsspannung und weiteres Abfangen von Spannungseinbrüchen aufgrund von Leistungsspitzen verwendet.

\paragraph{Spannungsteiler:}
Der Spannungsteiler (100 kOhm; 23,9 kOhm) am FB Pin sorgt für die richtige Ausgangsspannung von 5 V.
Dieser Spannungsteiler kann frei über die folgende Formel konfiguriert werden:
\[R_{FBB}=\frac{R_{FBT}}{\frac{V_{OUT}}{V_{REF}}-1}\]
Im Datenblatt des Spannungsreglers sind bereits Werte für die am häufigsten verwendeten Spannungen vorgegeben.

\paragraph{Kühlung:}
Der Spannungsregler muss gekühlt werden, da bei ihm eine nicht vernachlässigbare Verlustleistung auftritt.
Unter dem \ac{ic} ist eine große Massenfläche bereitgestellt, die zur Kühlung dient.
Die Kühlfläche des Spannungsreglers wird mit Durchkontaktierungen (Vias) mit der Massefläche auf der Unterseite der Platine verbunden und somit gekühlt.

\section{Lautsprecher-Verstärker}
Ein Audio-Leistungsverstärker (oder Leistungsverstärker) ist ein elektronischer Verstärker, der elektronische Audiosignale mit geringer Leistung, wie z.B. das Signal von einem Radioempfänger oder einem E-Gitarren-Tonabnehmer, auf einen Pegel verstärkt, der hoch genug ist, um Lautsprecher oder Kopfhörer anzutreiben.
Während das Eingangssignal eines Audio-Leistungsverstärkers, wie z.B. das Signal einer elektrischen Gitarre, vielleicht nur einige hundert Mikrowatt misst, kann seine Ausgangsleistung bei kleinen Geräten der Unterhaltungselektronik, wie z.B. Radiowecker, einige zehn oder hundert Watt für eine Heimstereoanlage, mehrere tausend Watt für die Beschallungsanlage eines Nachtclubs oder Zehntausende von Watt für ein großes Rockkonzert-Soundsystem betragen. \cite[vgl.][]{electronicshub-pwm-amp}

\subsection{Anforderungen}
Wichtige Entwurfsparameter für Audio-Leistungsverstärker sind Frequenzgang, Verstärkung, Rauschen und Verzerrung.
Diese sind voneinander abhängig; eine Erhöhung der Verstärkung führt oft zu einer unerwünschten Zunahme von Rauschen und Verzerrung.
Eine negative Rückkopplung reduziert zwar die Verstärkung, aber auch die Verzerrung.
Der Verstärkerchip soll einen Lautsprecher mit 8 \tOmega\ betreiben können.
Er sollte eine Leistung aufweisen, damit die Töne beim Lautsprecher ausreichend laut abgespielt werden.
Der Lautsprecherverstärker soll einen möglichst hohen Wirkungsgrad besitzen, um die Verlustleistung möglichst niedrig zu halten und damit zusätzliche Kühlung zu vermeiden.

\subsection{Auswahl}
Der PAM8403 ist ein 3W, Klasse-D-Audioverstärker. Er bietet einen niedrigen THD+N-Wert (bedeutet „total harmonic distortion plus noise“.
Sie wird normalerweise, durch Eingabe einer Sinuswelle, Bandsperrfilterung des Ausgangs und Vergleich des Verhältnisses zwischen dem Ausgangssignal mit und ohne Sinuswelle, gemessen), wodurch eine hochwertige Klangwiedergabe erreicht wird. 
Die neue filterlose Architektur ermöglicht es dem Gerät, den Lautsprecher direkt anzusteuern, so dass keine Tiefpassausgangsfilter erforderlich sind, wodurch Systemkosten und Leiterplattenfläche eingespart werden.
Der PAM8403 weist eine Effizient von bis zu 90 \% auf, deutlich höher als Verstärker aus den anderen Verstärkerklassen.
Diese weisen in den niederen Leistungsbereich meist eine Effizient von 50 \% auf.
Bei einem 8 \tOmega\ Lautsprecher ist eine maximale Leistung von 1.5 W möglich und ein Wirkungsgrad von 90 \%.

\subsection{Funktionen}
\paragraph{Maximale Verstärkung:}
Der PAM8403 hat zwei interne Verstärkerstufen.
Die Verstärkung der ersten Stufe ist extern konfigurierbar, während die der zweiten Stufe intern fest eingestellt ist.
Die Regelverstärkung der ersten Stufe wird durch die Wahl des Verhältnisses von RF zu RI eingestellt, während die Verstärkung der zweiten Stufe auf 2x festgelegt ist.
Der Ausgang von Verstärker 1 dient als Eingang zu Verstärker 2, sodass die beiden Verstärker Signale erzeugen, die in der Größe identisch sind, sich aber in der Phase um 180° unterscheiden.

\paragraph{Mute Funktion:}
Der MUTE-Pin ist ein Eingang zur Steuerung des Ausgangszustandes des PAM8403.
Ein logisches Low an diesem Pin deaktiviert die Ausgänge, und ein logisches High an diesem Pin aktiviert die Ausgänge.
Dieser Pin kann als schnelle Deaktivierung oder Aktivierung der Ausgänge ohne Lautstärkeschwund verwendet werden.
Der MUTE-Pin kann aufgrund des internen Pull-ups potentialfrei bleiben.

\paragraph{Shutdown Funktion:}
Um den Stromverbrauch bei Nichtbenutzung zu reduzieren, enthält der PAM8403 eine Abschaltung, mit der die Vorspannungsschaltung des Verstärkers abgeschaltet werden kann.
Diese Abschaltfunktion schaltet den Verstärker ab, wenn am SHDN-Pin ein logischer Low-Wert anliegt.
Durch Schalten des mit GND verbundenen SHDN-Pins wird die Stromaufnahme des PAM8403 im Leerlauf minimiert.
Der SHDN-Pin kann aufgrund des internen Pull-ups potentialfrei bleiben.

\subsection{Zusatzbeschaltung}
Der Verstärker benötigt nur wenige Komponenten als Zusatzbeschaltung.
Die Spannungsversorgung wird mit Kondensatoren parallelgeschaltet, um mögliche Spannungsschwankungen auszugleichen.
Der Eingang des Audiosignales vom \ac{rpi} wird zuerst über einen Kondensator geführt um Störungen zu vermeiden und Gleichanteile herauszufiltern.\par

Der Audioverstärker hat insgesamt 3 Masseanschlüsse, darunter befindet sich ein Masseanschluss für die interne Steuerschaltung.
Die zwei anderen Pins sind für die zwei Lautsprecher selbst, über sie fließt der Strom des Lautsprechers ab.

\subsection{Kühlung}
Der Audioverstärker muss nicht aktiv gekühlt werden, da ein Wirkungsgrad von rund 90 \% vorliegt.
Das bedeutet bei einer Leistung von 1.5 W tritt nur eine Verlustleistung 0.15 W auf.

\section{Mikrocontroller}
Ein Mikrocontroller ist ein Halbleiterchip, der einen Prozessor und zugleich auch Peripheriefunktionen enthält.
Der Arbeits- und Programmspeicher befindet sich vollständig auf demselben Chip, deshalb ist der Mikrocontroller ein Ein-Chip-Computersystem.
Auf modernen Controllern finden sich auch komplexe Peripheriefunktionen, sowie die Schnittstellen \ac{usb}, \ac{i2c} und \ac{spi}, \ac{pwm}-Ausgänge oder Analog-Digital-Umsetzer.
Es gibt eine breite Menge an verschiedenen Mikrocontrollern.
Angefangen bei Controllern mit einem Takt von 4 kHz und Leistung von wenigen Milliwatt oder Mikrowatt für einfache Aufgaben, wie Warten auf Betätigen eines Tasters.
Bis zu Controllern mit einem Takt von mehrere MHz und einen Programmspeicher von mehreren kBytes für größere und komplexere Aufgaben.\par

Der AVR-Mikrocontroller ATmega328P wurde aus mehreren Gründen ausgewählt:
\begin{itemize}
	\item Das Programmieren dieses Chips wird einerseits durch die Unterstützung der Arduino \ac{ide-dev}, mit der wir bereits sehr gut vertraut sind, sehr stark vereinfacht.
	\item Das Hochladen des Programmes in den ATmega328P stellt ebenfalls kein Problem dar, ein Arduino kann als \ac{icsp} verwendet werden.
	\item Der Controller benötigt nur eine minimale Beschaltung, um ordnungsgemäß zu funktionieren.
\end{itemize}

\subsection{Atmel AVR}
Die AVR-Mikrocontroller von Atmel (jetzt Microchip Technology Inc.) sind wegen ihrer übersichtlichen internen Struktur, der In-System-Programmierbarkeit, und der Vielzahl von kostenlosen Programmen zur Softwareentwicklung (Assembler, Compiler) beliebt. Diese Eigenschaften und der Umstand, dass viele Typen in einfach handhabbaren \ac{dip}-Gehäusen verfügbar sind, machen den AVR zum idealen Mikrocontroller für Anfänger.
Über die Bedeutung des Namens \enquote{AVR} gibt es verschiedene Ansichten; manche meinen es sei eine Abkürzung für Advanced Virtual \ac{risc}, andere vermuten dass der Name aus den Anfangsbuchstaben der Namen der Entwickler (Alf Egil Bogen und Vegard Wollan \ac{risc}) zusammengesetzt wurde. Laut Atmel ist der Name bedeutungslos.

\subsection{Architektur}
Die Architektur ist eine 8-Bit-Harvard-Architektur, das heißt, es gibt getrennte Busse zum Programmspeicher (Flash-ROM, dieser ist 16 bit breit) und Schreib-Lese-Speicher (RAM). Der Programmcode kann ausschließlich aus dem Programmspeicher ausgeführt werden. Weiterhin sind die Adressräume unabhängig (d.h. beide Speicher besitzen eigene Adressbereiche, die sich wertemäßig überschneiden können). Bei der Programmierung in Assembler und einigen C-Compilern bedeutet dies, dass sich Konstanten aus dem ROM nicht mit dem gleichen Code laden lassen wie Daten aus dem RAM. Abgesehen davon ist der Aufbau des Controllers recht übersichtlich und birgt wenige Fallstricke.
\begin{itemize}
\item 32 größtenteils gleichwertige Register
\item davon 1–3 16-bit-Zeigerregister (paarweise)
\item ca. 110 Befehle, die meist 1–2 Taktzyklen dauern
\item Taktfrequenz bis 32 MHz
\item Betriebsspannung von 1,8 – 5,5 V
\item Speicher (1-256 kB Flash-ROM, 0-4 kB EEPROM, 0-16 kB RAM)
\item Peripherie: Analog-Digital-Wandler 10 Bit, 8- und 16-Bit-Timer mit \ac{pwm}, \ac{spi}, \ac{i2c}, \ac{uart}, Analog-Komparator, Watchdog
\item 64 kB externer SRAM (ATmega128, ATmega64, ATmega8515/162)
\item JTAG bei den größeren ATmegas
\item debugWire bei den neueren AVRs		
\end{itemize}
\cite[Architektur]{mikrocontroller-avr}

\subsection{Funktion}
Der Mikrocontroller übernimmt in diesem Projekt mehrere Aufgaben.
\subsubsection{Watchdog-Timer für Raspberry Pi}
Ein Watchdog-Timer überwacht, ob ein System noch funktionsfähig ist, oder ob es irgendwo festhängt.
In letzterem Fall wird vom Watchdog-Timer in den meisten Anwendungen ein Hardware-Reset ausgelöst.\par

Der Mikrocontroller kommuniziert periodisch mit dem \ac{rpi} über \ac{uart}, d.h. der Mikrocontroller sendet eine Nachricht und der \ac{rpi} schickt eine Antwort zurück.
Wenn die Antwort des \ac{rpi} ausbleibt, hat sich dieser offensichtlich aufgehängt und muss zurückgesetzt werden.
Der Mikrocontroller setzt den Reset-Pin des \ac{rpi} auf Low und erzwingt damit einen Neustart der Station.
Nachdem der \ac{rpi} neugestartet ist, meldet er dies dem Mikrocontroller.
Der Mikrocontroller geht wieder in den Zustand des Überwachens über und der \ac{rpi} setzt seine normale Funktion fort.

\subsubsection{IO-Handling}
Über den Mikrocontroller werden sämtliche digitale Eingänge aufgenommen und an den \ac{rpi} weitergegeben.
Dies ist für die Diebstahlsicherung der Außenstation essenziell.
Weiters steuert der \ac{rpi} über die Digital-Ausgänge des Mikrocontrollers den Audio-Verstärker-\ac{ic} hinsichtlich Betrieb an.

\subsection{Zusatzbeschaltung}
\paragraph{Reset-Pin:}
Der Reset Pin des Mikrocontroller ist bei einer Spannung von 0 V aktiv.
Damit er im normalen Betrieb nicht aktiviert wird, wird er über einen Widerstand auf 5 V gezogen.
Um den Reset Pin nun zu aktivieren, muss nur zwischen dem Widerstand und dem Reset Pin 0 V angelegt werden.
Damit verändert sich das Potential des Reset Pin auf 0 V und wird aktiviert.

\paragraph{Clock:}
Die Clock des Mikrocontrollers wird durch einen Schwingquarz erzeugt.
Ein Schwingquarz, häufig vereinfachend abgekürzt als Quarz bezeichnet, ist ein elektronisches Bauelement, welches zur Erzeugung von elektrischen Schwingungen mit einer bestimmten Frequenz dient.
Über die Schwingfrequenz wird die Arbeitsgeschwindigkeit des Mikrocontrollers festgelegt.
In diesem Fall wird ein Quartz mit 6 MHz verwendet.