\wip
\section{Konzept}
Die neue Lösung sollte den Aufbau einer Station vereinfachen und die Ausfallrate einer Station auf gänzlich null reduzieren.
Um dieses Ziel zu erreichen werden folgende Ansätze definiert:

\paragraph{Zusammenfassen der Platinen zu einer einzigen:}
Die Reduktion der Anzahl der Platinen auf eine Platine ermöglicht die Zusammenfassung von verschiedenen Modulen.
Damit werden die Produktionskosten gesenkt und die Montage in den Stationen vereinfacht.

\paragraph{Reduzierung der Bauteile:}
Die Zusammenfassung der Module auf eine einzelne Platine ermöglicht die Reduktion von Bauteilen.
Insbesondere werden Bauteile für die Verbindungstechnik eingespart.
Dies führt zu einer Kostenreduktion bei gleichzeitiger Senkung der Fehlermöglichkeiten. 

\paragraph{Entfall der Verdrahtung zwischen den Einzelplatinen:}
Der Entfall der Verdrahtung führt zu einer Steigerung der Übersichtlichkeit und reduziert zusätzlich den Platzbedarf.
Aufwände für die händische Verdrahtung und die Prüfung der Verbindungen der einzelnen Platinen entfällt.
Fehlverdrahtung wird konstruktiv durch die Einzelplatine verhindert.

\paragraph{Verbesserung der Spannungsversorgung:}
Leistungsbeschränkungen sowie die Verwendung eines neuen verbesserten Spannungsreglers verbessern die Spannungsqualität.
Die verstärkte Spannungsversorgung ermöglicht einen späteren update auf den leistungsstärkeren RPi Model 4.

\paragraph{Einführung WatchDogTimer:}
Zusätzlich wird ein Watchdog mittels eines Mikroprozessors verbaut.
Dieser wird benötigt, um eventuelle Ausfälle des RPi und somit der gesamten Station zu erkennen.
Falls ein Fehler auftritt setzt der Mikroprozessor den RPi zurück und die Station startet neu.
Nachdem die Station neugestartet ist, wird der Normalbetrieb wieder aufgenommen und der Watchdog nimmt den Zustand des Überwachens wieder ein.

%Lösungen für probleme, zusätzliche vorteile
\section{Spannungsversorgung}
\subsection{Anforderungen}
Um einen möglichst reibungslosen Betrieb zu garantieren, werden an die Spannungsversorgung mehrere Anforderungen gestellt:
\begin{itemize}
	\item Der Eingangsspannungsbereich muss mindestens 24 bis 35 V betragen.
	\item Die gewünschte Ausgangsspannung beträgt 5 V.
	\item Der Oberwellenanteil (ripple) der Ausgangsspannung soll unter 1\% liegen. 
	\item Die Strombelastbarkeit muss mindestens 2.5 A betragen.
\end{itemize}
Die Versorgungsspannung wird über PoE (Power over Ethernet) zur Verfügung gestellt.
Der Spannungsregler muss diese Art der Versorgungsspannung unterstützen.
Der Spannungsregler sollte die 24 V auf 5 V möglichst verlustarm reduzieren.
Alle versorgten Komponenten auf der Platine arbeiten mit einer Spannung von 5 V.
Diese Komponenten sind z.B. der RPi, der Mikrocontroller und die Verstärkerschaltung für Mikrofon und Lautsprecher.\par

Die Spannungsversorgung soll bei Spannungsschwankungen kurzzeitig die Versorgung aufrechterhalten, um dem Versagen einer Station vorzubeugen.
Zur Stützung der Spannungsversorgung werden Kondensatoren auf der Eingangsspannungsseite und auf der Ausgangsspannungsseite benötigt.

\subsection{Auswahl Spannungsregler}
Aufgrund der oben angeführten Anforderungen wird der Spannungsregler LM33630 ausgewählt.
Dieser Spannungsregler kann im Bereich von 3.8 V bis 36 V betrieben werden.
Die Ausgangsspannung ist immer kleiner als die Eingangsspannung und liegt im Bereich von 1 V bis 24 V.
Der LM33630 ist ein Step-down Spannungsregler, das bedeutet er kann nur von einer höheren auf eine niedrigere Spannung regeln.\par

Um eine Schwingung, die wenig Aufwand zur Glättung benötigt, zu erzeugen verwendet der Spannungsregler eine hohe Schaltfrequenz von 2.1 MHz.
Der LM33630 hat standardmäßig einen maximalen Ausgangsstrom von 3 A, diese wird auch nahezu vollständig benötigt für den RPi, der bezieht allein bereits ca. 1.5 A und der Lautsprecher mit der Vorschaltung benötigt auch über 0.5 A.\par

Vom LM33630 sind mehrere Varianten verfügbar, es gibt die DDA und RNX Variante.
Die DDA Variante hat 8 Pins und ein eingebauter Kühlkörper, der mit einen großen Massefeld verbunden werden kann, damit der IC nicht überhitzt.
Die RNX Variante hat 12 Pins und anstatt eines größeren Kühlfläche mehrere Ground Pins, die ebenfalls zur Kühlung des ICs dienen.
Zusätzlich gibt es eine weitere Unterteilung bei den zwei Varianten, es gibt verschiedene Schaltfrequenzen und diese sind 400, 1400 und 2100 kHz.\par
Die technischen Daten dieses Spannungsreglers sind in vollem Umfang im Datenblatt ersichtlich. \cite[vgl.][]{lmr33630-datasheet}

\subsection{Funktion des Spannungsreglers}
Ein Step-down Spannungsregler arbeitet mit einen PWM Signal, das anschließend mittels einer Induktivität geglättet wird.\par
Bei der Pulsweitenmodulation (engl. Pulse Width Modulation, abgekürzt PWM) wird das Verhältnis zwischen der Einschaltzeit und Periodendauer eines Rechtecksignals bei fester Grundfrequenz variiert. Das Verhältnis zwischen der Einschaltzeit tein und der Periodendauer T=tein+taus wird als das Tastverhältnis p bezeichnet. (laut DIN-IEC 60469-1: Tastgrad) (engl. Duty Cycle, meist abgekürzt DC, nicht zu verwechseln mit Direct Current = Gleichstrom). \cite[Einleitung]{mikrocontroller-spannungsregler}\par

Über einen Spannungsteiler kann die Ausgangsspannung variiert werden.
Weiters kann der Spannungsregler abgeschaltet werden, wenn 0 V am Enable Pin angelegt werden.
Dies sorgt für eine komplette Abschaltung der Steuerlogik.
Die Verwendung dieser Funktion ist jedoch nicht vorgesehen.

\subsection{Zusatzbeschaltung des Spannungsreglers}
Der Spannungsregler ist universell einsetzbar, weshalb die genauen Betriebseigenschaften über die externe Beschaltung festgelegt werden.
\paragraph{Eingangskondensator:}
Auf der Eingangsseite, d.h. der 24 V Seite wird ein 820 µF Kondensator verbaut.
Dieser dient in erster Linie zum Abfangen und Ausgleichen von Spannungsschwankungen.
Diese Spannungsschwankungen können aufgrund von sich schlagartig ändernden Leistungsverbräuchen entstehen.

\paragraph{Ausgangskondensator:}
Nach dem Spannungsregler auf der 5 V Seite werden zwei Kondensatoren mit jeweils der Kapazität von 1 mF verbaut, das ergibt eine Gesamtkapazität auf der 5 V Seite von 2 mF.
Diese werden zur nahezu vollständigen Glättung der Ausgangsspannung und weiteres Abfangen von Spannungseinbrüchen wegen Leistungsspitzen verwendet.

\paragraph{Spannungsteiler:}
Der Spannungsteiler (100 kOhm; 23,9 kOhm) am FB Pin sorgt für die richtige Ausgangsspannung von 5 V.
Dieser Spannungsteiler kann frei über die folgende Formel konfiguriert werden:
\[R_{FBB}=\frac{R_{FBT}}{\frac{V_{OUT}}{V_{REF}}-1}\]
Im Datenblatt des Spannungsreglers sind bereits Werte für die am häufigsten verwendeten Spannungen vorgegeben.

\paragraph{Kühlung:}
Der Spannungsregler muss gekühlt werden, da bei ihm eine nicht vernachlässigbare Verlustleistung auftritt.
Unter dem IC ist eine große Massenfläche bereitgestellt, die zur Kühlung dient.
Die Kühlfläche des Spannungsreglers wird mit Durchkontaktierungen (Vias) mit der Massefläche auf der Unterseite der Platine verbunden und somit gekühlt.

\section{Mikrofon-Vorverstärker}

\section{Lautsprecher-Verstärker}
\subsection{Anforderungen}
\subsection{Auswahl}
\subsection{Funktionen}
\subsection{Zusatzbeschaltung}
Der Verstärker benötigt nur wenige Komponenten als Zusatzbeschaltung.
Die Spannungsversorgung wird mit Kondensatoren parallelgeschaltet, um mögliche Spannungsschwankungen auszugleichen.
Der Eingang des Audiosignales vom RPi wird zuerst über einen Kondensator geführt um Störungen zu vermeiden und Gleichanteile herauszufiltern.\par

Der Audioverstärker hat insgesamt 3 Masseanschlüsse, darunter befindet sich ein Masseanschluss für die interne Steuerschaltung.
Die zwei anderen Pins sind für die zwei Lautsprecher selbst, über sie fließt der Strom des Lautsprechers ab.

\section{Mikrocontroller}
Ein Mikrocontroller ist ein Halbleiterchip, der einen Prozessor und zugleich auch Peripheriefunktionen enthält.
Der Arbeits- und Programmspeicher befindet sich vollständig auf demselben Chip, deshalb ist der Mikrocontroller ein Ein-Chip-Computersystem.
Auf moderne Controller finden sich auch komplexe Peripheriefunktionen, sowie USB, I²C, SPI, PWM-Ausgänge oder Analog-Digital-Umsetzer.
Es gibt eine breite Menge an verschiedenen Mikrocontrollern.
Angefangen bei Controllern mit einem Takt von 4 kHz und Leistung von wenigen Milliwatt oder Mikrowatt für einfache Aufgaben, wie abwarten auf Betätigen eines Tasters.
Bis zu Controllern mit einem Takt von mehrere MHz und einen Programmspeicher von mehreren kBytes für größere und komplexere Aufgaben.\par

Der AVR-Mikrocontroller ATMega328P wurde aus mehreren Gründen ausgewählt:
\begin{itemize}
	\item Das Programmieren dieses Chips wird einerseits durch die Unterstützung der Arduino IDE, mit der wir bereits sehr gut vertraut sind, sehr stark vereinfacht.
	\item Das Hochladen des Programmes in den ATMega328P stellt ebenfalls kein Problem dar, ein Arduino kann als In-System-Programmer (ISP) verwendet werden.
	\item Der Controller benötigt nur eine minimale Beschaltung, um ordnungsgemäß zu funktionieren.
\end{itemize}

\subsection{Atmel AVR}
Die AVR-Mikrocontroller von Atmel (jetzt Microchip Technology Inc.) sind wegen ihrer übersichtlichen internen Struktur, der In-System-Programmierbarkeit, und der Vielzahl von kostenlosen Programmen zur Softwareentwicklung (Assembler, Compiler) beliebt. Diese Eigenschaften und der Umstand, dass viele Typen in einfach handhabbaren DIL-Gehäusen (DIP) verfügbar sind  machen den AVR zum idealen Mikrocontroller für Anfänger.
Über die Bedeutung des Namens "AVR" gibt es verschiedene Ansichten; manche meinen es sei eine Abkürzung für Advanced Virtual RISC, andere vermuten dass der Name aus den Anfangsbuchstaben der Namen der Entwickler (Alf Egil Bogen und Vegard Wollan RISC) zusammengesetzt wurde. Laut Atmel ist der Name bedeutungslos.

\subsection{Architektur}
Die Architektur ist eine 8-Bit-Harvard-Architektur, das heißt, es gibt getrennte Busse zum Programmspeicher (Flash-ROM, dieser ist 16 bit breit) und Schreib-Lese-Speicher (RAM). Programmcode kann ausschließlich aus dem Programmspeicher ausgeführt werden. Weiterhin sind die Adressräume unabhängig (d.h. beide Speicher besitzen eigene Adressbereiche, die sich wertemäßig überschneiden können). Bei der Programmierung in Assembler und einigen C-Compilern bedeutet dies, dass sich Konstanten aus dem ROM nicht mit dem gleichen Code laden lassen wie Daten aus dem RAM. Abgesehen davon ist der Aufbau des Controllers recht übersichtlich und birgt wenige Fallstricke.
\begin{itemize}
\item 32 größtenteils gleichwertige Register
\item davon 1–3 16-bit-Zeigerregister (paarweise)
\item ca. 110 Befehle, die meist 1–2 Taktzyklen dauern
\item Taktfrequenz bis 32 MHz
\item Betriebsspannung von 1,8 – 5,5 V
\item Speicher (1-256 kB Flash-ROM, 0-4 kB EEPROM, 0-16 kB RAM)
\item Peripherie: AD-Wandler 10 Bit, 8- und 16-Bit-Timer mit PWM, SPI, I²C (TWI), UART, Analog-Komparator, Watchdog
\item 64 kB externer SRAM (ATmega128, ATmega64, ATmega8515/162); (Bei den XMEGAs bis zu 16 MB (128 Mbit) externer SDRAM)
\item JTAG bei den größeren ATmegas
\item debugWire bei den neueren AVRs		
\end{itemize}
\cite[Architektur]{mikrocontroller-avr}

\section{Funktion}
Der Mikrocontroller übernimmt in diesem Projekt mehrere Aufgaben.
\subsection{Watchdog-Timer für Raspberry Pi}
Ein Watchdog-Timer überwacht, ob ein System noch funktionsfähig ist, oder ob es irgendwo festhängt.
In letzterem Fall wird vom Watchdog-Timer in den meisten Anwendungen ein Hardware-Reset ausgelöst.\par

Der Mikrocontroller kommuniziert periodisch mit dem RPi über UART, d.h. der Mikrocontroller sendet eine Nachricht und der RPi schickt eine Antwort zurück.
Wenn die Antwort des RPi ausbleibt, hat sich dieser offensichtlich aufgehängt und muss zurückgesetzt werden.
Der Mikrocontroller setzt den Reset-Pin des RPi auf Low und erzwingt damit einen Neustart der Station.
Nachdem der RPi neugestartet ist, meldet er dies dem Mikrocontroller.
Der Mikrocontroller geht wieder im Zustand des Überwachens über und der RPi setzt seine normale Funktion fort.

\subsection{IO-Handling}
Über den Mikrocontroller werden sämtliche digitale Eingänge aufgenommen und an den RPi weitergegeben.
Dies ist für die Diebstahlsicherung der Außenstation essenziell.
Weiters steuert der RPi über die Digital-Ausgänge des Mikrocontrollers den Audio-Verstärker-IC hinsichtlich Betrieb an.

\section{Zusatzbeschaltung}
\paragraph{Reset-Pin:}
Der Reset Pin des Mikrocontroller ist bei einer Spannung von 0 V aktiv.
Damit er im normalen Betrieb nicht aktiviert wird, wird er über einen Widerstand auf 5 V gezogen.
Um den Reset Pin nun zu aktivieren, muss nur zwischen dem Widerstand und dem Reset Pin 0 V angelegt werden.
Damit verändert sich das Potential des Reset Pin auf 0 V und wird aktiviert.

\paragraph{Clock:}
Die Clock des Mikrocontroller wird durch einen Schwingquarz erzeugt.
Ein Schwingquarz, häufig vereinfachend abgekürzt als Quarz bezeichnet, ist ein elektronisches Bauelement, welches zur Erzeugung von elektrischen Schwingungen mit einer bestimmten Frequenz dient.
Über die Schwingfrequenz wird die Arbeitsgeschwindigkeit des Mikrocontrollers festgelegt.
In diesem Fall wird ein Quartz mit 6 MHz verwendet.