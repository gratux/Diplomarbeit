\section{Vergleich}
Für die App-Programmierung stehen viele verschiedene Rahmenwerke bereits zur Verfügung.
Dieser Abschnitt beschäftigt sich mit dem Vergleich der einzelnen Möglichkeiten, sowie einer endgültigen Auswahl eines der Rahmenwerke zur Verwendung im Diplomprojekt.

\paragraph{Microsoft Xamarin}
ist ein Rahmenwerk, mit dem man Multiplatform-Applikationen für unter Anderem Android, iOS, UWP und noch viele weitere Platformen erstellen kann.
Es basiert auf Microsoft's .NET-Rahmenwerk und dem Mono-Projekt, welches sich als Ziel gesetzt hat, das .NET-Rahmenwerk auf andere Platformen zu portieren.
Xamarin bietet mehrere Projekttypen an, darunter Xamarin.Android und Xamarin.iOS für native App-Entwicklung und Xamarin.Forms für großteils platformunabhängige Entwicklung.
Eine Xamarin.Forms-Solution besteht daher aus drei Teilen: einem Portable/.NET-Standard-Projekt und je ein natives Xamarin-Projekt für jede Zielplatform.
Der Vorteil Xamarin's ist, dass der Großteil des geschriebenen Codes im platformunabhängigen Projekt bleibt und nur für wenige Funktionen auf native Programmierung zurückgegriffen wird, wie zum Beispiel für hardwarenahe Audio-Aufnahme.
Außerdem ist das Xamarin-Projekt Open Source, das bedeutet jeder kann den Quellcode betrachten und unter Umständen Verbesserungen vorbringen.

\paragraph{Apache Cordova}
ist ein Appentwicklungs-Rahmenwerk welches ursprünglich von der Firma Nitobi entwickelt wurde.
Im Jahr 2011 wurde Nitobi vone Adopbe aufgekauft und das Rahmenwerk wurde zu PhoneGap umgetauft.
Später wurde eine quelloffene Version unter dem ursprünglichen Namen veröffentlicht.
Apps werden mittels gängiger Webtechnologie, wie zum Beispiel JavaScript, CSS und Ähnlichen, entwickelt und realisiert, weshalb das Rahmenwerk alle gängigen Platformen unterstützt.
Der Vorteil von Cordova liegt im enormen Support dieser Webtechnologien, jedoch sind Sprachen wie JavaScript eher weniger für Back-End-Programmierung geeignet.
Noch dazu kommt, dass an der HTL Anichstraße großteils die Programmierung nur in C und C\# gelehrt wird, welshalb die Entwicklung mit JavaScript nicht realistisch ist.

\paragraph{Andere}
Neben Apache Cordova existieren noch viele weitere Rahmenwerke, die wie Cordova auf Webtechnologie aufbauen, um so die platformunabhängigkeit garantiern zu können. Allerdings bringen diese alle dieselben Probleme wie Cordova mit sich.

\section{Build-Umgebung}
Aufgrund obiger Aufstellung wurde Visual Studio 2019 der Firma Microsoft als Entwicklungsumgebung gewählt.
Diese steht für Schüler und Private gratis auf der Seite \href{https://visualstudio.microsoft.com/}{https://visualstudio.microsoft.com/} zum Download zur Verfügung.
%versionsvergleich hier https://visualstudio.microsoft.com/vs/compare/
Visual Studio ist das offizielle Werkzeug für die Programmierung mit dem Xamarin-Rahmenwerk.

%Betriebssystem, Hardwareanforderungen
%
\subsection{Installation Visual Studio} %nur wenn noch Text gebraucht wird
Die minimalen Hardware-, bzw. Software-Anforderungen können der \href{https://docs.microsoft.com/en-us/visualstudio/releases/2019/system-requirements}{Microsoft-Doku\-men\-tation des Visual Studio} entnommen werden:

\begin{figure}[H]
    \centering\begin{tabular}{ | r | p{0.6\linewidth} l | }
        \hline
        Unterstützte Betriebssysteme
        & - Windows 10, Version 1703 oder höher (LTSC und S werden nicht unterstützt)\\
        & - Windows Server 2019\\
        & - Windows Server 2016\\
        & - Windows 8.1 (mit Update 2919355)\\
        & - Windows Server 2012 R2 (mit Update 2919355)\\
        & - Windows 7 SP1 (mit neuesten Windows-Updates)\\
        & \\
        Hardware
        & - 1,8-GHz-Prozessor oder schneller; Quad-Core oder besser empfohlen\\
        & - 2 GB RAM; 8 GB RAM empfohlen\\
        & - Speicherplatz auf der Festplatte: Mindestens 800 MB, je nach installierten Features bis zu 210 GB des verfügbaren Speicherplatzes. Eine typische Installation erfordert 20–50 GB freien Speicherplatz.\\
        & - Festplattengeschwindigkeit: Zur Verbesserung der Leistung installieren Sie Windows und Visual Studio auf einem Festkörperlaufwerk (SSD).\\
        & - Grafikkarte, die eine Auflösung von mindestens 720p (1280 x 720) unterstützt. Visual Studio funktioniert am besten mit einer Auflösung von WXGA (1366 x 768) oder höher.\\
        \hline
    \end{tabular}
    \caption{Soft- und Hardware-Anforderungen Visual Studio}    
\end{figure}

%Quellenangabe!!
%Anforderungen für Android Emulator
% Der Android-Emulator hat zusätzliche Anforderungen, die über die grundlegenden Systemanforderungen für Android Studio hinausgehen, die im Folgenden beschrieben werden:

%     SDK-Tools 26.1.1 oder höher
%     64-Bit-Prozessor
%     Fenster: CPU mit UG-Unterstützung (uneingeschränkter Gast)
%     HAXM 6.2.1 oder höher (HAXM 7.2.0 oder höher empfohlen)

% Der Einsatz von Hardwarebeschleunigung stellt unter Windows und Linux zusätzliche Anforderungen:

%     Intel-Prozessor unter Windows oder Linux: Intel-Prozessor mit Unterstützung für Intel VT-x, Intel EM64T (Intel 64) und Execute Disable (XD)-Bit-Funktionalität
%     AMD-Prozessor unter Linux: AMD-Prozessor mit Unterstützung für AMD-V-Virtualisierung (AMD-V) und Supplemental Streaming SIMD Extensions 3 (SSSE3)
%     AMD-Prozessor unter Windows: Android Studio 3.2 oder höher und Windows 10. April 2018 oder höher für die Windows Hypervisor-Plattform (WHPX)

% Um mit Android 8.1 (API-Level 27) und höheren Systembildern arbeiten zu können, muss eine angeschlossene Webcam die Fähigkeit haben, 720p-Frames aufzunehmen.

%Vorsicht be\item Unterstützte Betriebssystemei der Installation

Auf der oben genannten Website steht der Visual Studio 2019 Installer zum Download bereit.
Hier ist bereits die geünschte Version, in diesem Fall die Community-Edition, auszuwählen.
\begin{figure}[H]
    \centering\includegraphics[width=0.9\linewidth]{images/auswahl_rahmenwerk/download.png}    
    \caption{Auswahl der Visual-Studio-Edition}
\end{figure}
Nach dem Download und anschließender Installation können die gewünschten Programmteile ausgewählt werden.
Für die Entwicklung von Xamarin-Applikationen ist das "Mobile development with .NET''-Pakted erforderlich.
\begin{figure}[H]
    \centering\includegraphics[width=0.9\linewidth]{images/auswahl_rahmenwerk/installation.png}    
    \caption{Auswahl des Xamarin-Rahmenwerks bei der Installation}
\end{figure}

Keine weiteren Einstellungen oder Auswahlen sind zwingend erforderlich, allerdings empfielt es sich noch zusätzlich das ,,.NET desktop development''-Paket zu installieren.
Mit diesem kann man neue Technologien des .NET Frameworks in einer einfacheren Konsolenanwendung ausprobieren und kennenlernen.
\begin{figure}[H]
    \centering\includegraphics[width=0.9\linewidth]{images/auswahl_rahmenwerk/installation2.png}    
    \caption{Zusatzpaket für Desktop-Entwicklung mit .NET}
\end{figure}
Nach Fertigstellung der Installation von Visual Studio 2019 erscheint dieses am Desktop und im Startmenü als neues Icon.

\subsection{Installation Android SDK}
Das Xamarin-Rahmenwerk benötigt zum Entwickeln von Android-Apps noch zusätzlich mehrere Teile des Android SDK der Firma Google. Mit diesem wird während dem Kompilieren das Programm in ein installierbares Android-Packet, kurz APK, umgewandelt.
Damit dies richtig funktioniert, müssen alle Komponenten, auf die im Programm zugegriffen wird, über den Android-SDK-Manager installiert werden.
%Android SDK komponente
\begin{figure}[H]
    \centering\includegraphics[width=0.9\linewidth]{images/auswahl_rahmenwerk/android_sdk_installation.png}    
    \caption{Android SDK Manager}
\end{figure}
\begin{figure}[H]
    \centering\includegraphics[height=15cm]{images/auswahl_rahmenwerk/android_sdk_auswahl.png}    
    \caption{Notwendige Teile des Android SDK}
\end{figure}
Besonders wichtig sind hier die Pakete der Google Play Services und die Goole Cloud Messaging Library, welche für das Empfangen von Push-Benachrichtung zwingend erforderlich sind. Diese Benachrigtung wird in einem eigenen Kapitel näher erklärt.
Die Android SDK Tools sind für jede Art von Android-Applikation mit Xamarin erforderlich.
Bei Bedarf kann über dieses Menü auch der offizielle Android-Emulator installiert werden.
%System images

Nach dem Start von Visual Studio muss eine neue ,,Solution'' angelegt werden. Diese beinhaltet alle Programmteile, welche für die Xamarin-Entwicklung benötigt werden. 

\begin{figure}[H]
    \centering\includegraphics[width=0.9\linewidth]{images/auswahl_rahmenwerk/installation2.png}    
    \caption{Anlegen einer neuen Solution}
\end{figure}





%Weg vom Programm zur binary


\section{Gedanken}
% Als Programmiersprache wird in diesem Projekt die Sprache C verwendet. Die
% Programmentwicklung in C ist im Vergleich zur Entwicklung in der Assemblersprache schneller und einfacher möglich. Trotzdem bietet die Sprache C die
% Möglichkeit eines sehr hardwarenahen Zugriffs auf die verfügbare Peripherie des
% Controllers durch Zugriffsmöglichkeit auf die einzelnen Register.
% 
% Ziel der Build-Werkzeuge ist es somit, den C-Programmcode in Maschinencode
% zu übersetzen und diesen in das Intel-Hex-Format überzuführen.
% Zur Bewältigung dieser Aufgabe kann unter Windows entweder auf die offizielle Entwicklungsumgebung Atmel-Studio zurückgegriffen werden (vgl. Atmel Studio 7 2016) oder es wird das WinAVR-Softwarepaket verwendet, das eine wesentlich schlankere Installation bietet und einen Editor, den GCC-Compiler und
% alle weiteren benötigten Programme zur Erstellung des Hex-Files aus dem CProgrammcode mitliefert (vgl. Homepage WinAVR 2016). Das Programm kann
% von der Website direkt als Installer-Datei heruntergeladen werden.
% WinAVR liefert das Programm Programmer’s Notepad mit, das bereits alle benö-
% tigten Makros zur Übersetzung des Programms besitzt. Es reicht, eine beliebige
% Datei aus dem Ordner der Mikrocontrollersoftware mit Programmer’s Notepad zu
% öffnen und in der Menüleiste unter Tools die Auswahl Make all anzuklicken. Im
% Ordner wird eine Datei namens main.hex erzeugt, die nun beispielsweise mithilfe
% der Software des Programmieradapters zum Mikrocontroller übertragen werden
% kann.
% Unter Linux existieren bei den meisten Distributionen vorgefertigte Pakete unter
% dem Namen avr-gcc bzw. gcc-avr, welche die benötigte Software beinhalten. Sollte das Paket avr-libc nicht automatisch mitinstalliert werden, muss es manuell
% installiert werden. Zur Übersetzung des Projekts kann in der Kommandozeile in
% den Ordner mit den Quelldateien gewechselt und dort der Befehl make all ausgeführt werden. Der Befehl make clean löscht alle erzeugten Dateien, sodass nur
% mehr die Quelldateien übrigbleiben und make program überträgt das Programm
% zum Mikrocontroller. Zur Übertragung ruft make die Software avrdude auf, die
% eine Vielzahl verschiedener Programmiergeräte unterstützt. Um das verwendete
% Programmiergerät einzustellen, müssen in der Datei Makefile im Ordner mit den
% Quelldateien die Optionen AVRDUDE_PROGRAMMER und AVRDUDE_PORT abgeändert
% werden. Die Dokumentation des Programms avrdude liefert weitere Informationen
% über die unterstützten Programmer (vgl. AVRDUDE Documentation 2016).
% Achtung Der Mikrocontroller im Projekt wird mit einer Betriebsspannung von
% 3 V anstatt mit 5 V versorgt. Viele Programmer gehen jedoch davon aus, dass
% der Mikrocontroller mit 5 V betrieben wird. Vor dem Programmiervorgang sollte
% der Programmer auf 3 V-Betrieb umgestellt werden. Die genaue Vorgehensweise dazu muss der jeweiligen Dokumentation des Programmiergeräts entnommen
% werden.
% Fusebits Einige Grundfunktionen des Mikrocontrollers können über die sogenannten Fusebits eingestellt werden. Diese Bits werden, anders als die Konfigu-
% 1057. Auswahl Mikrocontroller
% rationsbits in gewöhnlichen Registern, nicht durch die Software am Controller,
% sondern direkt mithilfe des Programmieradapters eingestellt (vgl. ATmega 644
% Microcontroller Datasheet 2012, Seite 285 ff.). Für dieses Projekt müssen die Fusebits zur Einstellung der Taktquelle (CKSEL3..0) auf den Wert 0b1101 gestellt
% werden. Diese Einstellung steht für einen Low power crystal oscillator mit einer
% Frequenz zwischen 3 und 8 MHz. Außerdem ist es wichtig, das JTAG-Interface
% mithilfe des Bits JTAGEN (→ auf 0 setzen) zu deaktivieren, da die entsprechenden Pins des JTAG-Interfaces, das zum Debuggen verwendet werden kann, für
% andere Aufgaben benötigt werden. Möchte man die Mikrocontrollersoftware testen, bietet es sich an, das Bit EESAVE auf 0 zu setzen. Dadurch wird der Inhalt
% des EEPROMs nicht bei jedem Programmiervorgang gelöscht. So gehen die Einstellungen der Software nicht jedes Mal verloren, wenn ein Fehler in der Software
% ausgebessert wird.
% Zur Programmierung der Fusebits kann meist die entsprechende Software des
% Programmieradapters verwendet werden. Das Atmel-Studio und avrdude bieten
% die Funktionalität ebenfalls. Für avrdude bietet sich als komfortableres Frontend
% beispielsweise das Programm avr8-burn-o-mat an, das für alle gängigen Betriebssysteme auf der Projektwebsite heruntergeladen werden kann (vgl. AVR8 BurnO-Mat - a GUI for avrdude 2016).
% Wichtig Atmel bezeichnet Fusebits als programmed, wenn sie auf 0 gesetzt und
% als unprogrammed, wenn sie auf 1 gesetzt sind. Viele Programme zum Einstellen
% der Fusebits bieten eine Checkbox für die Einstellung an, die angehakt werden
% muss, um das Bit auf 0 zu setzen. Das kann verwirrend sein. In diesem Fall ist
% Vorsicht geboten. Die Dokumentation der entsprechenden Software sollte konsultiert werden, da eine falsche Einstellung der Taktquelle dazu führen kann, dass der
% Controller nicht mehr reagiert und zur „Wiederbelebung“ ein Taktsignal mittels
% Funktionsgenerator eingespeist werden muss.

%microsoft VS2019
%frei zugänglich
%nur auf windows, kein linux-support
%...