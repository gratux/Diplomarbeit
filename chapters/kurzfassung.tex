\section{Deutsch}
Die vorliegende Diplomarbeit beschäftigt sich mit der Erweiterung einer \ac{rpi}-basierten Videosprechanlage um eine Smartphone-Applikation, sowie der grundlegenden Überarbeitung der Stations-Hardware.
Zusätzlich soll ein Linux-basierter, aus dem Internet erreichbarer Server zum Verteilen der Video-Streams eingerichtet werden.
%Für die Entwicklung der Smartphone-App wird das Xamarin-Framework herangezogen, sodass die Applikation sowohl auf Android-, als auch iOS-Geräten lauffähig ist.
\par
Dieses Projekt basiert auf einer früheren Diplomarbeit von Sebastian Wagner und Tobias Pfluger an der HTL Anichstraße aus dem Jahr 2017/18, welche ein voll funktionsfähiges Videosprechanlagensystem hervorbrachte.
Die verwendete Hardware bietet Verbesserungspotential und wird im Zuge dieser DIplomarbeit weiterentwickelt.
\par
Die Hardware-Erweiterung besteht grundsätzlich aus zwei Teilen: zum einen werden die vielen Elektronik-Bausteine in einer zentralen Platine vereint, was eine einfachere und kostengünstigere Fertigung erlaubt.
Zusätzlich wird die aktuelle Hardware der Station um einen Watchdog-Timer erweitert, welcher im Fehlerfall die Anlage zurücksetzt.
\par
Softwareseitig wird die Anlage um eine Smartphone-App ausgebaut, mit welcher der Fernzugriff auf das System ermöglicht werden soll.
\par
\newpage

\begin{otherlanguage}{british}
	\section{English}
	This diploma thesis deals with the extension of a \ac{rpi}-based video intercom system by a smartphone application, as well as the fundamental revision of the station hardware.
	In addition, a Linux-based server, accessible from the Internet, will be set up to distribute the video streams.
	%The Xamarin framework will be used for the development of the smartphone app, so that the application can be run on both Android and iOS devices.
	\par
	This project is based on an earlier diploma thesis by Sebastian Wagner and Tobias Pfluger at the HTL Anichstraße from the year 2017/18, which produced a fully functional video intercom system.
	The hardware used offers potential for improvement and will be further developed in the course of this diploma thesis.
	\par
	The hardware extension basically consists of two parts: on the one hand, the many electronic components are combined in a central circuit board, which allows easier and more cost-effective production.
	In addition, the current hardware of the station is extended by a watchdog timer, which resets the system in case of an error.
	\par
	On the software side, the system will be extended by a smartphone app, which will enable remote access to the system.
	\par
\end{otherlanguage}