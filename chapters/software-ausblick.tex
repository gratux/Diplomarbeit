\wip
\section{Gesicherte Verbindung}% VPN-Zugriff
\section{Weitere mobile Plattformen}
Neben dem ausprogrammierten Android-System gibt es noch das weit verbreitete iOS der Apple-Geräte.
Mit Xamarin.Forms lässt sich dieses verglichen mit anderen Entwicklungsmethoden einfach unterstützen.
Es muss lediglich der plattformspezifische Teil auf die neue Plattform portiert werden, während das portable Projekt unverändert bleibt.
Dies bedeutet, dass analog zum PiBell.Android ein Projekt PiBell.iOS geschrieben werden muss, um die Plattform zu unterstützen.
In diesem Projekt müssen alle Funktionen des Android-Teils repliziert werden.
Wichtig für die Entwicklung ist auch, dass die Endgeräte zum Übertragen und Testen verfügbar sind.
Im Fall der iOS-Plattform wird ein Apple Mac für die Übertragung und ein Apple iPhone zum Testen benötigt.\par

Für Testzwecke kann die Applikation auf bis zu fünf iPhones installiert und getestet werden.
Wenn die App schlussendlich veröffentlicht werden soll, ist eine kostenpflichtige Mitgliedschaft beim Apple Developer Program notwendig.
Diese beläuft sich auf etwa USD 99 pro Jahr \cite[vgl.][Integrated Development Environment (IDE) Availability]{msdoc-xamarin-fundamentals}
%\blindtext