Jedes moderne Wohngebäude ist inzwischen mit einer Gegensprechanlage ausgestattet.
Eine solche Anlage werlaubt es dem Wohnungsbesitzer mit jemandem vor der Haustür zu reden, bevor dieser hereingelassen wird.
Manche Wohnungen besitzen bereits eine Videosprechanlage, die nicht nur den Ton, sondern zusätzlich ein Video von außen dem Bewohner bereitstellt.
Dem Gast wird jedoch immer noch nur der Ton von innen übertragen.\par

Im Zuge einer früheren Diplomarbeit an der HTL Anichstraße entwickelten \_ und \_ eine Videogegensprechanlage, welche mehrere Innenstationen und die bidirektionale Video- und Tonübertragung unterstützt.
Die Idee ist, dass in einem Wohnungskomplex pro Partei eine Innestation verbaut wird, sowie eine Außenstation am Hauseingang.
Die einzelnen Innenstationen, also Wohnungsparteien, können dann nicht nur mit der Außenstation, sondern auch mit anderen Innenstationen per Video und Ton kommuniziernen.
Die Stationen wurden mit Hilfe eines \ac{rpi} realisiert, um die Kosten gering zu halten.\par

Am Anfang des 5. Schuljahres bat uns unser FI-Professor DI(FH) Mario Prantl an, dieses Projekt durch eine Smartphone-App und eine Hardware-Überarbeitung zu verbessern.
Für die Entwicklung der App sollte das Xamarin-Rahmenwerk verwendet werden, damit die Applikation auf sowohl Android-, als auch iOS-Geräten lauffähig ist. Der Benutzer soll über die App das Video der entsprechenden Station abrufen und mit dem Gesprächspartner sprechen können.
Die Hardware-Überarbeitung hat mehrere Ziele; zum Einen soll dadurch die komplexität des internen Aufbaus einer Station verringert werden.
Zum Anderen soll mit Hilfe eines Watchdog-Timers das Langzeit-Betriebsverhalten verbessert werden.
