% Im Juni 2000 stellte das Unternehmen Microsoft das \ac{dotnet} vor.
% Kurz darauf wurde von Miguel de Icaza das sogenannte Mono-Projekt als Open-Source gestartet, welches eine Linux-Version des \ac{dotnet} darstellen soll.
% Am 16. Mai 2011 kündigt Miguel de Icaza an, dass das Mono-Projekt vom Unternehmen Xamarin weiterentwickelt wird, wobei einige wichtige Mitglieder des Mono-Entwickler-Teams weiterhin daran beteiligt sind.

% Das Unternehmen Xamarin wurde mit der Absicht gegründet, Software auf mobile Geräte zu vertreiben. 

\section{Übersicht}
Grundsätzlich besteht das Software-Projekt aus drei Teilen:
\paragraph{PiBell} ist das portable Porjekt, auch .NET-Standard-Projekt genannt.
Dieses beinhaltet den ganzen platformunabhängigen Code, sowie die Definition der UI-Oberfläche mittels XAML-Dateien.\par
\paragraph{PiBell.Android} beinhaltet allen Code, der platformspezifisch für Android geschrieben wurde. Unter anderem ist hier enthalten die Android-Implementation des Mikrofon-Auf\-nahme-Services, sowie das Berechtigungs-Management.\par
\paragraph{PiBell.iOS} beinhaltet wie PiBell.Android den platformspezifischen Code für die iOS-Plat\-form. Aufgrund mangelnder Entwicklungswerkzeuge wurde dieser Teil nicht großartig behandelt.\par
Der immer wieder auftretende Name PiBell ist dabei eine freigeistliche Erfindung. Der Begriff setzt sich zusammen aus RaspberryPi und dem englischen Wort für Glocke oder Klingel (Bell).\par
\begin{figure}
\tikzstyle{mybox} = [draw=black, fill=blue!20, very thick,
    rectangle, rounded corners, inner sep=10pt, inner ysep=20pt]
\begin{tikzpicture}
    %Grafik Teile der Solution ...
\end{tikzpicture}
\caption{Aufbau der Solution}
\end{figure}

%besteht aus drei teilen
%- portable project/.net standard project
%   platformunabhängiger teil, oberfläche, ...
%- Android project
    %android-spezifischer teil, berechtigungen, ...
%- iOS project
    %iOS-spezifischer Teil, ...
%project-name PiBell -> erfindung, leitet sich aus raspberry und Türglocke ab
%
\section{Portable Project}
\section{PiBell.Android}
\section{PiBell.iOS}